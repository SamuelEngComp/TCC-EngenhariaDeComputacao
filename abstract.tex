\chapter*{Abstract}


\noindent Bid Data is the term used to define large volumes of data, which relational databases could not store in a structured way, given that most data from BigData are non-structured type. In order to store such data, it was created a \textit{framework} Apache Hadoop, which has two essential tools for BigData storage and processing, which are: MapReduce e HDFS. Knowing that large technology companies, such as Facebook, Google e Amazon, make use of BigData, that is, utilize data from their own users, so they can direct marketing specific advertisements for each group of users that share similar interests, and doing so, generate revenue and profit, there was a concernment regards the data safety, which raises the following question: are these data accordingly protected against attacks? With this matter in mind, this paper proposes some good security practices in order to keep the data safe and 'to armor' the Hadoop environment, looking for possible faults and proposing pertinent safety practices to prevent theft and loss of data. The applied methodology consists of an scanning though a virtualized environment with three test scenarios, on which are verified the vulnerabilities that were found at the Hadoop environment, as well as how to rectify them.

\vspace{1.5ex}

{\bf Keywords}: Big Data, Security information, Database, java programming language, frameworks.